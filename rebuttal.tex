\documentclass[11pt,a4paper]{article}
\usepackage{times}
\usepackage{fullpage}
\setlength{\parindent}{0cm}


\begin{document}
\newcommand{\comment}[1]{\textit{``\ldots #1''}\par\vspace{0.5em}}
\newcommand{\response}[1]{#1\vspace{1em}}

\fbox{\textbf{Reviewer 1}}

\comment{when the authors state 'together with an even larger number
  of annotations', they shold explain what are annotations and how
  much is the large number;} 
\response{We have updated the text on page 1 to be more explicit in
  terms of what annotations are and provide a concete example of the
  sizes involved.}

\comment{For an overview I would expect a more comprehensive set of
  references.}
\response{We agree with the reviewer that a more comprehensive list of
references would be appropriate and useful. However, due to format
limitations of the article, we have been restricted to the use of XXX
references. To alleviate this problem, we have included a supplement
that provides a more extensive list of references, specifically
including a number of texts covering cheminformatics broadly.}

\fbox{\textbf{Reviewer 2}}

\comment{The authors mention limitations of SMILES implementations,
  but do not mention canonicalization in this context, often using the
  Morgan algorithm published in 1965 in J. Chem. Doc.}
\response{We have updated the text on page 3 to note the Morgan
  algorithm in the context of chemical structure representation}

\comment{The authors state that 3D information is lost when only
  considering the molecular graph. However, they do not make it clear
  that the 3D space is implicitly encoded in the molecular graph –
  this is essential to cover properly.}
\response{We have update the text on page 2 to note that 3D
  information is implicit in 2D representations}

\comment{If this manuscript were to be published it would be
  completely rewriting the history of cheminformatics}
\response{While we agree with the reviewer that some key milestones in
  the history of cheminformatics were not included in the original
  article, we believe that the addition of the supplemental history
  addresses this problem. Furthermore, our original aim was \emph{not}
  to provide a complete historical overview of the field. Others have
  already provided such overviews. Our aim was to present a brief
  overview of \emph{topics} in cheminformatics, selected because we
  believed that they represent on going challenges for the field and
  thus amenable to coss-disciplinary efforts.}

\fbox{\textbf{Reviewer 3}}

\comment{page 1, column 1, line 24: the authors need to provide a
  reference for the assertion that cheminformatics is an older field,
  particularly given the comments later in this
  manuscript.}
\response{We have removed the assertion that cheminformatics is older
  than bioinformatics. Primarily, because there is no obvious
  reference that claims this. But alse because, depending on what one
  considers bioinformatics and cheminformatics, one field can be
  identified as being older or younger than the other. We feel that
  claims on age do not strengthen the paper and have thus removed it}

\comment{p1, c1, l29-37: I would dispute that until recently the
  cheminformatics techniques have been closely guarded
  secrets. J. Chem. Doc. was founded in 1961 and other techniques even
  older, such as the Wiener index (1947), Wiswesser Line Notation
  (1949), etc. Not to mention the actual foundations of what we call
  cheminformatics back to the atomistic theory of the 19th century.}
\response{We have the updated the text on page 1 to attenuate this
  statement, stressing on the fact that much of the data in
  cheminformatics is proprietary, whereas only some of the techniques (such
  as SMILES canonicalization) are. Regarding the statement that atomistic
  theory laid the foundations of cheminformatics - this is certainly
  true in a very broad sense. For that matter the atomistic theory
  laid the ground for computational chemistry in general. In the
  supplemental history document, we make a distinction between
  computational chemistry and cheminformatics}

\comment{p1, c1, l32: I think the phrase miracle molecule could be
  more usefully replaced with new drugs or new small molecule
  therapeutics.}
\response{The text has been updated to use the term ``therapeutic
  molecule''}

\comment{p2, c2, l57: it might be worthwhile here clarifying in the text
precisely the properties that require satisfaction to deliver a small
molecule therapeutic, not ?promising?. A drug must be safe and
efficacious ultimately; perhaps this should be mentioned first
followed by the typical pitfalls and how they are assessed?}
\response{The text has been updated to provide some examples of
  properties that would characterize a therapeutically useful small molecule}

\comment{P3, c1, l41: the authors need to cover the Morgan algorithm
  (published in 1965) here and explain the canonicalization
  process. The issues mentioned in different canonicalization
  implementations providing different SMILES strings is also a
  challenge with InChI codes with different softwares giving different
  representation. Therefore, this is still not a solved problem as
  suggested here. Could the authors clarify this in the text?}
\response{We have updated the text to reference the Morgan
  algorithm. Regarding the issue of InChI codes, we believe that this
  is not the case. Currently, there is only one implementation of the
  InChI algorithm and even if alternative implementations were to be
  developed, the InChI specification is publically available. As a
  result, there should be no differing InChI representations for a
  given input structure.}

\comment{P4, c2, l17: citation needed on the size of chemistry space.}
\response{The appropriate reference has been added}

\comment{P4, c2, l23: the isomorphism problem has already been
  mentioned previously in this manuscript.}
\response{The text has been updated and simplified}

\comment{P4, c2, l41: the GDB-13 database contains 970 million
  molecules, which is nearly a billion, not a trillion.}
\response{The text has been updated to use the correct number}


\comment{P5, c1, l29: the normal phrase used to describe this concept
  is the similar property principle. The authors should also provide a
  reference.}
\response{This portion of the text has been restructured to refer to
  the similarity property principle and also include the relevant
  reference}

\comment{P5, c1, l40: QSAR should be referenced to Hansch et al. Also
  perhaps some discussion on what the authors mean by referring to
  these approaches as traditional.}
\response{We have updated text to include references to Hansch and
  Free \& Wilson. Regarding the use of ``traditional'', we believe the
  text explicitly explains why - the fact that QSAR as originaly
  defined only considers ligand features. However, we have rephrased it to use
  ``traditionally'' since one can argue that methods such as docking
  and pharmacophores are also QSAR methods, but consider both ligand
  and receptor.}

\comment{P5, c1, l41: I would dispute the assertion that these methods
  "ignore reception interactions" since the aim is to identify a
  correlation of biological response (e.g. pIC50) with chemical
  structure. The biological response is an explicit measurement of
  receptor interactions on the protein.}  
\response{We have updated
  the text to explicitly note that QSAR models do not usually take
  into account \emph{receptor features}. However we note that while
  receptor information is implicit in the $IC_{50}$, the value also
  includes other non-receptor related features such as
  permeability. Furthermore, lack of receptor information in QSAR
  models has been noted as the origin of activity cliffs (Guha and Van
  Drie, \textit{J.~Chem.~Inf.~Model.}, \textbf{2008}, \textit{48},
  1716--1728), thus supporting the statement that traditional QSAR
  models do indeed miss important information on receptor-ligand
  interactions.}

\comment{P5, c2, l22: I would mention naive Bayesian classifiers as well as
this is perhaps the most widely applied method in the field.}
\response{We have included the Na\"{i}ve Bayes as well}

\comment{P5, c2, l42: formal citation for one of Hopfinger?s papers is
  required here.}
\response{The relevant reference has been added}

\comment{P5, c2, l58: this assertion is made with no evidence to back
  it up. Why should multi-target models be more reliable? It is not
  clear to me that they should.}\marginpar{\textbf{Needs more work}}
\response{We have rephrased this statement to be less dogmatic}

\comment{P7, c2, l25: the authors should go back even further in the
  history of cheminformatics here before DENDRAL. While this is an
  excellent example of crossover, it is not by any means the founding
  of our field. Aspects of mathematical chemistry should also be
  mentioned in this article, which date back even further, such as
  1894 with the publication of ?The Principles of Mathematical
  Chemistry? published by Helm.}
\response{As noted previously, we have included a supplemental
  document that lists some of the milestones in the history of
  cheminformatics. However, we note that the suggested reference by
  Helm is not specifically related to cheminformatics. Rather, it is a
  mathematical treatment of physical chemistry. Given that DENDRAL is
  referenced in the context of structure enumeration, we feel that
  inclusion of this reference would be somewhat irrelevant. We have
  included this reference in the supplemental history}

\comment{P7, c2, l51: why is chemically in inverted commas and not
  algorithmically? I would prefer inverted commas for neither.}
\response{Quotes have been removed}

\comment{P8, c1, l10: ELN's should be ELNs.}
\response{Corrected}
 
\end{document}