\documentclass[11pt,a4paper]{article}
\usepackage{times}

\setlength{\parindent}{0cm}


\begin{document}
\newcommand{\comment}[1]{\textit{``\ldots #1''}\par\vspace{0.5em}}
\newcommand{\response}[1]{#1\vspace{1em}}

\fbox{\textbf{Reviewer 1}}

\comment{when the authors state 'together with an even larger number
  of annotations', they shold explain what are annotations and how
  much is the large number;} 
\response{We have updated the text on page 1 to be more explicit in
  terms of what annotations are and provide a concete example of the
  sizes involved.}

\comment{For an overview I would expect a more comprehensive set of
  references.}
\response{We agree with the reviewer that a more comprehensive list of
references would be appropriate and useful. However, due to format
limitations of the article, we have been restricted to the use of XXX
references. To alleviate this problem, we have included a supplement
that provides a more extensive list of references, specifically
including a number of texts covering cheminformatics broadly.}

\fbox{\textbf{Reviewer 2}}

\comment{The authors mention limitations of SMILES implementations,
  but do not mention canonicalization in this context, often using the
  Morgan algorithm published in 1965 in J. Chem. Doc.}
\response{We have updated the text on page 3 to note the Morgan
  algorithm in the context of chemical structure representation}

\comment{The authors state that 3D information is lost when only
  considering the molecular graph. However, they do not make it clear
  that the 3D space is implicitly encoded in the molecular graph –
  this is essential to cover properly.}
\response{We have update the text on page 2 to note that 3D
  information is implicit in 2D representations}

\comment{If this manuscript were to be published it would be
  completely rewriting the history of cheminformatics}
\response{While we agree with the reviewer that some key milestones in
  the history of cheminformatics were not included in the original
  article, we believe that the addition of the supplemental history
  addresses this problem. Furthermore, our original aim was \emph{not}
  to provide a complete historical overview of the field. Others have
  already provided such overviews. Our aim was to present a brief
  overview of \emph{topics} in cheminformatics, selected because we
  believed that they represent on going challenges for the field and
  thus amenable to coss-disciplinary efforts.}

\fbox{\textbf{Reviewer 3}}

\end{document}