\documentclass{sig-alternate}

\usepackage{array}
\usepackage{pifont}
\usepackage{url}
\usepackage{graphicx}
\usepackage{multirow}

\newcommand{\none}{\ding{55}}
\newcommand{\least}{\ding{51}}
\newcommand{\little}{\ding{51}\ding{51}}
\newcommand{\lots}{\ding{51}\ding{51}\ding{51}}


\begin{document}
\pagenumbering{arabic}


\title{Open Questions for Computer Science and Cheminformatics}
\numberofauthors{9}
\author{
\alignauthor
Joerg Kurt Wegner\\
       \affaddr{Tibotec BVBA}\\
       \affaddr{Turnhoutseweg 30}\\
       \affaddr{2340 Beerse Turnhout, Belgium}\\
       \email{jwegner@its.jnj.com}
% 2nd. author
\alignauthor
Aaron Sterling\\
       \affaddr{Department of Computer Science}\\
       \affaddr{Iowa State University}\\
       \affaddr{Ames, Iowa, USA}\\
       \email{sterling@iastate.edu}
% 3rd author
\alignauthor
Rajarshi Guha\\
\affaddr{NIH Center for Translational Therapeutics}\\
\affaddr{9800 Medical Center Drive}\\
\affaddr{Rockville, MD 20850}\\
\email{guhar@mail.nih.gov}
}

\additionalauthors{Additional authors:
Andreas Bender (University of Cambridge, email: {\texttt{andreas.bender@cantab.net}}),
Jean-Loup Faulon (University of Evry, email: {\texttt{Jean-Loup.Faulon@issb.genopole.fr}}),
Janna Hastings (European Bioinformatics Institute, Cambridge, UK, email: {\texttt{janna.hastings@gmail.com}}),
Noel O'Boyle (University College Cork, Cork, Ireland, email: {\texttt{baoilleach@gmail.com}}),
John Overington (European Bioinformatics Institute, Cambridge, UK, email: {\texttt{jpo@ebi.ac.uk}}),
Herman Van Vlijmen (Tibotec, Beerse, Belgium, email: {\texttt{hvvlijme@its.jnj.com}}), and
Egon Willighagen (Karolinska Institutet, Stockholm, Sweden, email: {\texttt{egon.willighagen@ki.se}})
.}
\maketitle
%
We offer the following open questions to suggest concrete interdisciplinary research directions for computer scientists working with chemoinformaticians.  This list is by no means exhaustive.  We have focused on questions that, in our opinion, are answerable, and whose answers would be of great interest to the field.  The questions are grouped roughly by computer science subarea, and include both open theoretical problems and ``requests'' for open-source practical implementations.  We hope CS researchers new to cheminformatics find this useful.

\section*{Algorithmic graph theory}
\begin{enumerate}
\item \emph{Design an algorithm that approximately counts the number of (3D) conformers of a chemical formula.}

This question encompasses the open questions of \emph{approximately counting the number of stereoisomers, or the number of tautomers, of a chemical formula}.  Goldberg and Jerrum designed an algorithm which, given a chemical formula, would output an isomer of that formula chosen uniformly at random~\cite{RandomlySampling}.  Perhaps the main theoretical obstacle to overcome is that molecules tend to be \emph{chiral} (that is, they have a ``handedness'' or 3D orientation), whereas traditional graph theory (and the Goldberg/Jerrum algorithm) treats graphs with identical vertices and edges as isomorphic.  So part of the question could be rephrased as, ``Given a labeled vertex set, count the number of structures that are identical on that vertex set, except that they differ in their 3D orientation.''  Mathematical chemists have designed measures of molecular chirality~\cite{ChiralityMeasures}. A resolution of this problem may require connecting graph enumeration algorithms to knot theory or other topics in topology~\cite{TopologicalLook}.
%
\item \emph{Write code that approximately counts the number of conformers of a chemical formula.}

While this is related to the previous question, it is conceivable that one could answer this question without solving the theoretical problem.  For example, as we note in the main article, the commercial software MOLGEN can approximately enumerate structures that satisfy given spectral data, even though related theoretical enumeration problems appear wide open.
%
\item \emph{Design and implement efficient subgraph isomorphism algorithms for useful special cases of molecular graphs}.

The Subgraph Isomorphism Problem is known to be $\textsf{NP}$-complete (hence possibly harder than the Graph Isomorphism Problem, which is not known to be $\textsf{NP}$-complete).  Nevertheless, finding maximum common subgraphs to match chemical structures is of fundamental importance in chemistry; hence, much work has been invested in partial solutions to this problem.  Raymond and Willet reviewed the state of the art in 2002~\cite{MCSreview}.  As one possible way to attack this problem, we note the empirical fact that molecular graphs are of bounded degree, and are observed to have \emph{treewidth} $\leq 5$~\cite{treewidth}.  Bounded treewidth is at least theoretically useful~\cite{Epp-JGAA-99}, and it may be possible to improve on current open-source implementations whose isomorphism-checking routines are written for all graphs, instead of taking advantage of special properties of chemical graphs.

\item \emph{Searching within complex data types, e.g. molecules, for semantic web approaches}.

One key concept of the linked data web, the semantic web, is that different data sources can be readily integrated with each other. Still, in the field
of Cheminformatics, we are not only interest in linking two molecules 
(the linking normalization problem for different protomers, tautomers, or special cases of isomerisms remain open), but we
are also interested in being able to search efficiently within molecules when being linked via semantic web approaches. Typical
searches will require being able to apply substructure or similarity searches.
What could be algorithmic solutions for this?
\end{enumerate}

\section*{Cryptography}
One-way molecular featurization (???)

\section*{Data mining}
\begin{enumerate}
\item evaluation of similarities in a heterogeneous network (JKW: What are heterogenous networks?).  What is a specific example here?
\item integrate chemical structure information with ontologies.  Specific example problem?
\item Classify molecular descriptors up to equivalence, and dependence on one another.
\item systems-level understanding of small molecules (JKW: which systems? This will help being clearer on the challenge).  What does this mean, and what would a specific challenge problem be?
%\item \emph{Efficient molecule browsing, e.g. on scaffold level}.
%
%Chemical Abstract Services have a molecule browsing tool called SubScape, which allows to browse large-scale
%chemical spaces efficiently. What could be large-scale solutions for doing this within (combined and aligned) 
%public databases.
%
\item \emph{Dynamic similarity search on instant binary vectors}.

  Binary feature vectors are a common practice for chemical similarity
  searches. The typical process starts with 1. creating binary
  substructure (or other feature vectors) \cite{citeulike:8530538},
  2. creating fixed-length binary vectors of typically 1024 bits for
  reducing space requirements and speeding up further similarity
  searches (by loosing some accuracy), 3. creating further
  pre-computations for speeding up threshold based similarity searches
  \cite{doi:10.1021/ci800076s}.\\  Still, if we are interested to employ
  dynamic changes in the similarity encodings, e.g. using only a set
  of binary features, then previously done hashing or pre-computations
  might need to be redone efficiently on the fly. Finally, the major
  goal is to employ similarity searches \cite{doi:10.1021/ci200235e}
  on a scale of multiple million entries and more optimizations and
  benchmarking studies are urgently required, e.g. using GPUs
  \cite{doi:10.1021/ci1004948}, optimizing pair-wise similarity
  calculations \cite{MINF:MINF201100050}.
%

\item \emph{Large-scale browsing of molecular property spaces, e.g. on scaffold level, side-effect-level, ...}.

  Certain molecules might have hundreds of biological activities,
  side-effects in humans (SIDER database
  \cite{Kuhn_Campillos_Letunic_Jensen_Bork_2010}), or many other
  properties attached to them. What are large-scale mining and
  visulization options?  How can we mine private and public data
  sources at the very same time?
%
\item \emph{Chemical image/text mining in patents (curation)}.

  There are various tools for doing automatic text mining on chemical
  patents. Still, the overall acceptance rate of chemical text mining
  is improvable, since many medicinal chemists are very concerned
  about the data quality of such efforts.  What could be done to
  improve the mining quality, curate the obtained data, and to provide
  confidence level estimations for each molecule coming from patent
  mining? Do require image2structure and text2structure mining also
  data stores for ensuring a sufficient amount of confidence and data
  quality?  How can patent mining be used to create new drugs faster
  or to speed-up collaboration/licensing discussions?
\end{enumerate}

\section*{Machine learning}
\begin{enumerate}
\item \emph{Large-scale vectorial versus kernels molecule similarity}

  Vectorial molecule encodings can serve as efficient approximations
  of molecules.  Sometimes non-vectorial molecular 3D shape or
  molecule kernel comparisons might be more suitabe to compare
  molecules, since they might better correlate with activities. One
  key problem is that non-vectorial encodings require to compare all
  molecules (or their 3D conformational eplosions) in a pair-wise
  manner.  This becomes prohibitively expensive when considering
  millions of molecules.  Can dyadic data approaches help
  \cite{Hochreiter:2006:SVM:1159508.1159516}? Other approximations or
  cascading flows?
%
\item \emph{Using multiple annotations for improving molecular mining/predictions (chemogenomics)}

  As an example: Biological activities might not be independent of
  each other, but have a certain correlation between each other.  In
  Chemogenomics this is used for creating models of combining
  molecules with protein sequences, molecules with active sites of
  proteins, or molecules with biological activities of multiple
  assays. How can we optimize such highly complex mining scenarios,
  especially when considering large-scale data sets with hundred of
  thousands molecules and thousands of biological activities?  How can
  we combine, mine, and visualize categorial and continuous output
  variables, e.g. hydrophobicity of a molecule and toxicity in humans,
  by still being able to make concrete proposals to medicinal
  chemistry? Is analoging (creating very small modifications of a
  molecule and measuring its activities) really the most efficient way
  forward? If we test molecules, should we test it in a single
  biological assay or in multiple biological assays, if multiple,
  which ones?  If a company does not have a biological assay within
  reach, which other partner could offer testing a molecule within two
  days (vendor matching based on licenses or contracts)?
\end{enumerate} 

\section*{Software engineering}
\begin{enumerate}
\item \emph{Large-scale exchange and use of 3D conformational databases}

  There exist a multitude of programs generation 3D conformations of
  molecules. Still, often input formats vary and require to re-create
  3D conformational spaces with other programs. This does not only
  complicate the maintenance scenario of such databases, especially
  with the explosion of public databases, but also creates hugh space
  and time complexity issues when searching within such
  databases. What could be better interfaces, maintenance, data
  structures, and private/public sharing scenarios for conformational
  3D databases? 

\item \emph{Parallelization of cheminformatics problems \& algorithms}
Many cheminformatics tasks apply algorithms over large input files or
across many molecules. A simple way to parallelize this is to chunk
the input data and let individual threads/node process each chunk. A
trivial solution is to manually chunk the input and submit a series of
jobs to a scheduling system.

Can we employ modern frameworks like Hadoop to perform this? Can we
develop a generalized framework that includes the requisite
cheminformatics tools that allows users to seamlessly distribute jobs
over Hadoop and other map reduce systems?

Going further, are there cheminformatics problems that can make use of
the map/reduce paradigm at the algorithmic level? More generally, are
cheminformatics algorithms amenable to parallelization (that does not
involve chunking the input data)? Are they worth the effort?

\item \emph{Licensing in a parallel world} Many software vendors use
  different solutions for parallizing compute jobs: SGE, PVM, MPI,
  etc.  Is a cloud really an option?  What about SaaS with secured data
  transfer?  Can this also offer alternative licensing strategies for
  software suites in this domain?

\end{enumerate}

\section*{Enterprise software (KM,ELN)}
We know that the enterprise software and ELN market is still growing.
\begin{enumerate}
\item \emph{Public-private collaboration and security scenarios}

Let us assume an organization, e.g. a commercial company, has a single or a small number of established KM and ELN products. 
How can we improve the maintenance, leveraging, and collaboration with many external partners (each of them potentially
with another KM/ELN solution)? Which party is hosting which data in which data structure (ontologies?), and 
how can we ensure that only pre-defined data entries 
(and a limited number of annotations, e.g. biological activities) are visible to a partner.
How can this be organized for a multitude of partners? Cloud computing, user management, encryption granularity and efficient security management?
\end{enumerate}

\bibliographystyle{abbrv}
\bibliography{paper}
\end{document}